\subsection{Grading}
\begin{description}
\tightlist
\item[A (4.0)] Student \emph{exceeds} all expectations of the assignment. Evidence of exceeding expectations includes deeply engaging with the material, demonstrating of initiative beyond what can be expected by simply completing the assignment, nuanced understanding of the material, and fostering a collaborative environment for learning in and out of the classroom.
\item[B (3.0)] Student \emph{meets all} expectations of the assignment. \item[C (2.0)] Student \emph{meets most} expectations of the assignment.
\item[D (1.0)] Student \emph{meets some} expectations of the assignment.
\item[F (0.0)] Student \emph{fails to meet} expectations of the assignment.
\end{description}

\noindent Grades are given based on the grade points above. Final grades are determined by rounding the weighted average grade to two decimal places:

\vspace{.75\baselineskip}
\begin{center}
\begin{tabular}{*{4}{rl}}
\toprule & & 3.16 - 3.49 & B+ & 2.16 - 2.49 & C+ & & \\
3.84 - 4.00 & A & 2.84 - 3.15 & B & 1.84 - 2.15 & C & 0.50 - 1.49 & D \\
3.50 - 3.83 & A- & 2.50 - 2.83 & B- & 1.50 - 1.83 & C- & $<$0.50 & F \\ \bottomrule
\end{tabular}
\end{center}
\vspace{.5\baselineskip}