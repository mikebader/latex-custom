\subsection{Grading}
\begin{description}
\tightlist
\item[A (4.0)] Student \emph{exceeds} all expectations of the assignment. Evidence of exceeding expectations includes deeply engaging with the material, demonstrating of initiative beyond what can be expected by simply completing the assignment, nuanced understanding of the material, and fostering a collaborative environment for learning in and out of the classroom.
\item[B (3.0)] Student \emph{meets} expectations of all objectives identified for the assignment. Evidence of meeting expectations includes engaging with material, actively participating in classroom discussions, and basic understanding of the material.
\item[C (2.0)] Student \emph{meets some} expectations of the objectives identified for the assignment. See above for description of evidence of meeting expectations.
\item[D (1.0)] Student \emph{fails to meet most} expectations of objectives identified for the assignment. See above for description of meeting expectations.
\item[F (0.0)] Student \emph{fails to meet} expectations of objectives identified for the course. See above for description of meeting expectations.
\end{description}

\noindent Grades are given based on the grade points above. Final grades are determined by rounding the weighted average grade to two decimal places:

\vspace{.75\baselineskip}
\begin{center}
\begin{tabular}{*{4}{rl}}
\toprule & & 3.16 - 3.49 & B+ & 2.16 - 2.49 & C+ & & \\
3.84 - 4.00 & A & 2.84 - 3.15 & B & 1.84 - 2.15 & C & 0.50 - 1.49 & D \\
3.50 - 3.83 & A- & 2.50 - 2.83 & B- & 1.50 - 1.83 & C- & $<$0.50 & F \\ \bottomrule
\end{tabular}
\end{center}
\vspace{.5\baselineskip}